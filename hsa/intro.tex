\section{Introduction}
\label{introduction}

Performance in HPC has become an important factor, leading to complex
heterogeous platforms which include specialized devices such as GPU, FPGA, and
DSP. The use of specialized accelerators in HPC systems has grown rapidly after
processors hit the frequency barrier in early 2005. Since then HPC systems have
started using mulitple compute units in tandem to achieve the high performance.
Transistor densities have grown exponentially from continued innovations in
process technology. Over the past 40 years the mechanisms used to advance
micorprocessor performance has evolved, from executing a single instruction at a
time on a single processor to executing many parallel instructions on many
symmetric cores. With the continued integration of different processor cores,
accelerators and other processing elements heterogenous computing architectures
are on an exponential rise.
