% Copyright 2004 by Till Tantau <tantau@users.sourceforge.net>.
%
% In principle, this file can be redistributed and/or modified under
% the terms of the GNU Public License, version 2.
%
% However, this file is supposed to be a template to be modified
% for your own needs. For this reason, if you use this file as a
% template and not specifically distribute it as part of a another
% package/program, I grant the extra permission to freely copy and
% modify this file as you see fit and even to delete this copyright
% notice. 

\documentclass{beamer}

% There are many different themes available for Beamer. A comprehensive
% list with examples is given here:
% http://deic.uab.es/~iblanes/beamer_gallery/index_by_theme.html
% You can uncomment the themes below if you would like to use a different
% one:
%\usetheme{AnnArbor}
%\usetheme{Antibes}
%\usetheme{Bergen}
%\usetheme{Berkeley}
%\usetheme{Berlin}
%\usetheme{Boadilla}
%\usetheme{boxes}
%\usetheme{CambridgeUS}
%\usetheme{Copenhagen}
%\usetheme{Darmstadt}
%\usetheme{default}
%\usetheme{Frankfurt}
%\usetheme{Goettingen}
%\usetheme{Hannover}
%\usetheme{Ilmenau}
%\usetheme{JuanLesPins}
%\usetheme{Luebeck}
\usetheme{Madrid}
%\usetheme{Malmoe}
%\usetheme{Marburg}
%\usetheme{Montpellier}
%\usetheme{PaloAlto}
%\usetheme{Pittsburgh}
%\usetheme{Rochester}
%\usetheme{Singapore}
%\usetheme{Szeged}
%\usetheme{Warsaw}

\title{So is vim/emacs [in]efficient ?}

% A subtitle is optional and this may be deleted
\subtitle{emacs vs vim}

\author{Muhammad Falak R~Wani } 

\institute[IIIT-D] % (optional, but mostly needed)
{
	\inst{}
  Department of Computer Science\\
  IIIT-D
}

\date{ILUG-D, 2016}

\subject{Text Editors}

% If you have a file called "university-logo-filename.xxx", where xxx
% is a graphic format that can be processed by latex or pdflatex,
% resp., then you can add a logo as follows:

\pgfdeclareimage[height=0.5cm]{university-logo}{iiitd-logo.png}
\logo{\pgfuseimage{university-logo}}

% Delete this, if you do not want the table of contents to pop up at
% the beginning of each subsection:
\AtBeginSubsection[]
{
	\begin{frame}<beamer>{Agenda}
		\tableofcontents[currentsection,currentsubsection]
	\end{frame}
}

% Let's get started
\begin{document}

\begin{frame}
	\titlepage
\end{frame}

\begin{frame}{Agenda}
	\tableofcontents
	% You might wish to add the option [pausesections]
\end{frame}

% Section and subsections will appear in the presentation overview
% and table of contents.
\section{vim}

\subsection{History}

\begin{frame}{It all started with \textbf{ed} ...}{Then came \textbf{ex}}
	\textbf{do one thing, and do it well}\\ \pause
	It is not \textbf{VI} (roman \textbf{six}) but \textbf{`Wee Eyee'}.\\
	\pause
	\begin{itemize}
		\item {
				\textbf{ed}: The basic editor.
				\pause

			}
		\item {
				\textbf{sed}: The stream editor.
				\pause

			}
		\item {
				\textbf{ex}: Improved and rewritten ed.
				\pause
			}
		\item {
				\textbf{vi}: Visual mode in ex.
				\pause
			}
		\item {
				\textbf{vim}: Improved and rewritten vi.

			}
	\end{itemize}
\end{frame}

\subsection{Survival Skills}

% You can reveal the parts of a slide one at a time
% with the \pause command:
\begin{frame}{How to quit...}
	vim is a \textbf{modal} editor. \\ \pause
	You will benefit if you touch type.\\ \pause
	\begin{itemize}
		\item {
				\textbf{normal}: The default mode.
				\pause
				}
			\item {   
				\textbf{insert}: The insert mode.
				\pause
			}
			% You can also specify when the content should appear
			% by using <n->:
			%  \item<3-> {
		\item {
				\textbf{ex}: The ex mode.
				\pause
			}
	\end{itemize}
	How to get \alert{help }?\\ \pause
	In normal mode \textbf{:help}\\
	\alert{vimtutor} if you want an online tutorial
\end{frame}

\subsection{Power Usage}
\begin{frame}{Power Usage}{merely scratching the surface}
	vim is a very \textbf{powerful} editor \pause
	\begin{itemize}
			\item {
					\textbf{movement}: Awkward at first.
					\pause
					}
			\item {
					\textbf{windows}: Spliting windows and movement.
					\pause
					}
			\item {
					\textbf{tabs}: Moving around in tabs.
					\pause
					}
			\item {
					\textbf{indentation}: Indenting code.
					\pause
					}
			\item {
					\textbf{autocomplete}: Dial \textbf{x} for completion.
					\pause
					}
			\item {
					\textbf{compilation}: Find errors quickly. (\textit{colder, cnewer})
					\pause
					}
			\item {
					\textbf{man}: go to a specific manual page without leaving vim
					\pause
					}
			\item {
					\textbf{ctags}: move around code efficently.
					\pause
					}
			\item {
					\textbf{cscope}: yet another way of moving around code.
					}

	\end{itemize}
\end{frame}
\begin{frame}{Plugins}{extensibility - community}
	vim has a huge collection of third party plugins.. \pause
	\begin{itemize}
		\item {
				\textbf{pathogen}: A plugin for easy installation of other plugins
				\pause
				}
		\item {
				\textbf{NERDTree}: Explore your project.
				\pause
				}
		\item {
				\textbf{youcompleteme}: Autocompleteion.
				\pause
				}
		\item {
				\textbf{SuperTab}: Yet another Autocompleteion plugin.
				\pause
				}
		\item {
				\textbf{fugitive}: A git wrapper for vim
				\pause
				}
	\end{itemize}
	Easiest way to install is by \alert{pathogen}\\
	Just modify the vimrc \textbf{execute pathogen\#infect()}
\end{frame}

\section{emacs}

\subsection{History}
\begin{frame}{Written by \textbf{rms}...}{\textbf{Richard M Stallman}}
	\textbf{Do everything philosophy} \\
	\pause
	\begin{itemize}
		\item {
				\textbf{EMACS}: \alert{E}\textit{diting} \alert{MAC}\textit{ros}\alert{S}
				\pause

			}
		\item {
				Written in \textbf{C} and provides \textbf{lisp}.
				\pause

			}
		\item {
				It is self-documenting.
				\pause
			}
		\item {
				Church of Emacs
				\pause
			}
		\item {
				Editor War, the longest running flame war on internet. (who's the best)

			}
	\end{itemize}
\end{frame}

\subsection{Survival Skills}
\begin{frame}{How to quit...}
	It is not an \textbf{editor} but an\textbf{`Operating System'}, which happens to edit too.\\
	Emacs has \textbf{modes} which alter the meaning of commands. \\
	Emacs commands are elaborate sequence, called \alert{chords}. They are very confusing to startoff with.
	\begin{itemize}
		\item {
			\textbf{C-x}: The most used chords, they are used frequently.
			\pause
			}

		\item {   
			\textbf{M-x}: The extended chords, or functions typically.
			\pause

			}
		\item {
			\textbf{Major Mode}: What are you doing.
			\pause
			}
		\item {
			\textbf{Minor Mode}: Extension to the major-mode
			\pause
			}
	\end{itemize}
	How to get \alert{help }?\\ \pause
	emacs is a self documenting editor\\
	\textbf{C-h m} mode-specific help.\\
	\textbf{C-h f} describe a function.\\
	\textbf{C-h k} describe a function a key runs.\\
	\textbf{M-x help} interactive help.\\

\end{frame}

\subsection{Power Usage}
\begin{frame}{Power Usage}{merely scratching the surface}
	It takes some time to master emacs \pause
	\begin{itemize}
			\item {
					\textbf{movement}: Awkward at first.
					\pause
					}
			\item {
					\textbf{buffers}: Open files and \textit{not} files.
					\pause
					}
			\item {
					\textbf{windows}: Spliting windows and movement.
					\pause
					}
			\item {
					\textbf{frames}: Moving around in frames
					\pause
					}
			\item {
					\textbf{indentation}: Indenting code.
					\pause
					}
			\item {
					\textbf{autocomplete}: autocomplete-mode
					\pause
					}
			\item {
					\textbf{compilation}: Find errors quickly. (\textit{M-x compile})
					\pause
					}
			\item {
					\textbf{debugging}: Run a debugger inside emacs. (\textit{M-x gdb})
					\pause
					}
			\item {
					\textbf{man}: Go to a specific manual page without leaving emacs. (\textif{M-x woman})
					\pause
					}
			\item {
					\textbf{gtags}: Move around code efficently.
					\pause
					}
%			\item {
%					\textbf{cscope}: yet another way of moving around code.
%					}

	\end{itemize}
\end{frame}
\begin{frame}{Plugins}{extensibility - community}
	emacs has everything but the kitchen sink.\\ \pause
	\textbf{melpa} is the emacs repository for plugins.\\
	\begin{itemize}
		\item {
				\textbf{auto-complete}: auto completion. (builtin)
				\pause
				}
		\item {
				\textbf{company-mode}: better auto-complete.
				\pause
				}
		\item {
				\textbf{projectile}: View your projects.
				\pause
				}
		\item {
				\textbf{js2-mode}: javascript mode, best of the best.
				\pause
				}
		\item {
				\textbf{magit}: A git wrapper for emacs
				\pause
				}
		\item{
				\textbf{web-mode}: major mode for all html templates.
				\pause
				}
	\end{itemize}
	Installation is just calling a function \\
	\textbf{M-x package-list-packages}\\
\end{frame}
% Placing a * after \section means it will not show in the
% outline or table of contents.
\subsection{The Kitchen Sink}
\begin{frame}{What if..}{So what else can emacs do to impress me ?}
	As promised emacs has everything but the kitchen sink.\pause \alert{let's see} \\
	\begin{itemize}
		\item {
				\textbf{pdf}: Yes it can (builtin), but \textbf{pdf-tools} is better.
				\pause
				}
		\item {
				\textbf{ppt}: Yes it can (builtin)
				\pause
				}
		\item {
				\textbf{internet}: Yes, that too.
				\pause
				}
		\item {
				\textbf{shell}: Yes, it has a builtin shell.
				\pause
				}
		\item {
				\textbf{calender}: Yes, it manages your appointments.
				\pause
				}
		\item{
				\textbf{dairy}: It can keep your secrets too.
				\pause
				}
		\item{
				\textbf{music}: It so happens, it can do that too.
				\pause
				}
		\item{
				\textbf{notes}: \textbf{org-mode} is one of the single reasons people switch to emacs.
				\pause
				}
		\item{
				\textbf{games}: It has games too, if you are bored.
				\pause
				}

		\item{
				\textbf{mail}: It has a builtin mail client.
				\pause
				}
		\item{
				\textbf{irc}: It seem's it can do irc too.
				\pause
				}
		\item{
				\textbf{calculator}: It has a fully programable scientific calculator.
				\pause
				}

	\end{itemize}
	It seems for whatever reason, emacs people have made emacs do almost anything.\\
	They don't call it an operating system for just for fun.
\end{frame}
\section*{Summary}

\begin{frame}{Take Aways}
	\begin{itemize}
		\item
			Both \textbf{emacs \& vim} are \alert{very powerful} editors. 
		\item
			\alert{Invest} some time to learn them, you will not regret it.
		\item
			In appropriate hands both can \alert{shred} text at speed of thought.
		\item 
			Don't get involved in flame wars, exploit the best of both the worlds.
	\end{itemize}

	\pause
	\begin{itemize}
		\item
			I barely illustrated a mere 1\% of the power of these beasts. \pause
			\begin{itemize}
				\item
					Manuals are your friends. \pause
				\item
					I am feeling lucky, try it.
			\end{itemize}
	\end{itemize}
\end{frame}



% All of the following is optional and typically not needed. 
\appendix
\section<presentation>*{\appendixname}
\subsection<presentation>*{For Further Reading}

\begin{frame}[allowframebreaks]
	\frametitle<presentation>{For Further Reading}

	\begin{thebibliography}{10}

			\beamertemplatebookbibitems
			% Start with overview books.

		\bibitem{DN}
			Drew~Neil.
			\newblock {\em Practical Vim}.
			\newblock The Pragmaitc Programmers, 2012.


			\beamertemplatebookbibitems
			% Followed by interesting articles. Keep the list short. 

		\bibitem{RMS}
			Richard M~Stallman et al.
			\newblock {\em GNU Emacs Manual}.
			\newblock Free Software Foundation, 2014.
	\end{thebibliography}
\end{frame}

\end{document}


